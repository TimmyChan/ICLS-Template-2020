\documentclass{article}
\usepackage[utf8]{inputenc}

\title{     }
\author{     }
\date{     }

\addbibresource{library.bib}
\begin{document}

\section{General notes (Style "Heading 1" --- Arial, 12 pt, bold, first word capitalized)}
These are the formatting requirements for the papers that will be published in the CSCL 2017 proceedings. Your
paper must conform to these guidelines so that we can have uniform appearing proceedings. Time between
submission of the final copy and submitting them to the publisher is short, so submitting your paper following
these guidelines is necessary for insuring your paper’s inclusion in the proceedings. It is recommended that you
use the MS Word styles in this document as a template for your document.
\subsection{Body text formatting (Style “Heading 2” --- Arial, 12 pt, regular, first word capitalized)}
The initial paragraph following a header is in style “Body Text.” Use Times or Times New Roman 10 point.
“Full justify” paragraph text, i.e., left and right justify. Do not indent or leave a blank line following headings.
The margins for your entire paper should be 1 inch on all four sides, using A4 paper.

Subsequent paragraphs are in style “Body Text First Indent.” Do not leave blank lines between
paragraphs. Indent the first line of each paragraph .5 inches from the left margin.

\emph{DO NOT use page numbers, running heads, or footnotes.} If you must use notes, please use endnotes,
and place them immediately before the reference list. Do not use the Word processors automatic endpoint
features. Refer to endnotes in text using a standard full sized numeral inside parentheses \endnote{Place any endnotes after the main text of your paper, but before your reference list. Use Times or Times New Roman 9
pt text for your endnotes, and "full justify" the margins. Place the endnote number reference in parentheses in the left margin, using the same number as in the text of the paper.} without superscripting \endnote{Use style "Endnotes", with a .25 inch "hanging"; indent and a .25 inch tab setting, as in this example.}. For \emph{emphasis} in your text, use \emph{italics}.

Leave one line before each section header. This will be accomplished automatically if you used the MS
Word styles in this document. Do not use more than three levels of headers.

\subsubsection{Page length (Heading 3 --- Arial, 10 pt, regular, underlined, first word capitalized)}
{\color{red} Full papers may use a maximum of 8 pages; posters may use a maximum of 2 pages; short papers may use a maximum of 4 pages; symposia may use a maximum of 8 pages.} \emph{You may not exceed the maximum.} Refer to the
conference call for papers for details.

\subsubsection{Blind review for papers and posters}
Papers and posters will be reviewed blind, so it is important to prepare submissions so that reviewers do not
know the names of the authors. Please prepare your paper in a way that preserves author anonymity. 
{\color{red} \emph{Bulleted and enumerated lists are in 10 point Times New Roman font and having a 3 pt space before and after each line.
They are indented a quarter-inch on the left and have a quarter-inch hanging indent. }}
\begin{itemize}
\item Do not include author names and institutions under the title in your submissions.
\item Avoid using phrases like "our previous work" when referring to earlier publications by the authors.
\item Remove information that may identify the authors in the acknowledgements (e.g., grant IDs).
\item Avoid providing web links to sites or other supplementary materials that may identify the authors
\end{itemize}

\subsubsection{Extended quotes}
For extended quotes from source material, use Times or Times New Roman 10 pt, and indent the quote .5 inch
from both the left and right margins. "Full justify" the text for the extended quote. The extended quote should be
preceded and followed by one blank line.

\begin{quotation}
Lorem ipsum dolor sit amet, consectetaur adipisicing elit, sed do eiusmod
temporincididunt ut labore et dolore magna aliqua. Ut enim ad minim veniam, quis
nostrudexercitation ullamco laboris nisi ut aliquip ex ea commodo. (Euripedes, 1999, p. 213)
\end{quotation}

\subsubsection{Figures and tables}
All figures and tables must be referred to in your text (see Table 1). Color figures may be included, but they will be printed in the proceedings in black and white, so please be sure that they will reproduce acceptably without color (see Figure 1). All figures and tables should be centered. Table captions are underlined and aligned left
above the table. Figure captions are centered and placed below the figure.

\begin{table}[H]
\caption{An example of a table for the ICLS proceedings}

\centering
\begin{tabular}{c || c | c | c | c | c | c}
\hline
\multirow{2}{*}{index} & \multicolumn{2}{c|}{Item1}&\multicolumn{3}{c|}{Item2} & Item3 \\ \cline{2-7}
& Item1-1 & Item1-2 & Item2-1 & Item2-2 & Item 2-3 & Item 3-1 \\ \hline \hline 
1 & a & b & c & d & e & f \\ \hline
2 & g & h & i & j & k & l \\ \hline
\end{tabular}
\end{table}

\begin{figure}[H]
\begin{center}
\includegraphics[width=.25\textwidth]{logo.png}
\end{center}
\caption{The ISLS logo.}
\end{figure}

\subsubsection{Citation and reference style}
Use APA reference style throughout your paper and the reference section. Please refer to the Fifth Edition of the APA Publication Manual for full details and more extensive examples. Examples:

\begin{quotation}
\dots promoted with the use of technology \parencite[e.g.,][]{hawkins_tools_1987} \dots \\
\dots \textcite{lave_cognition_1988} moves the analysis of \dots \\
\dots Dewey (1929) called for systematic knowledge of teaching practice \dots \\
\dots orchestrating learning in naturalistic settings \parencite{brown_design_1992, bruner_postscript_1999}
\end{quotation}

\renewcommand{\notesname}{Endnotes (use Heading 1)} %edit this line to change the text of endnotes heading.
\theendnotes



\end{document}